% !TeX encoding = ISO-8859-1
% !TeX root = appunti.tex

\chapter{Prefazione}

\markboth{Prefazione}{Prefazione}

Questi Appunti nascono per esser la materia di una forma che si chiama \LaTeX. Il loro scopo iniziale era fornire sostanza per un tempo e un esercizio appena sufficienti per consentire al sottoscritto di cominciare a usare questo strumento nel modo pi� efficace e pulito possibile.

Per quanto riguarda la prima parte, i capitoli~\ref{cha:segnali}, \ref{cha:sistemi-lineari}, \ref{cha:sistemi-non-lineari}, \ref{cha:segnali-campionati} sono stati scritti a partire dalle lezioni del professor Martorella integrando successivamente dal~\citep{b:luise}. I capitoli~\ref{cha:segnali-periodici}, \ref{cha:segnali-aperiodici} sono stati invece ottenuti principalmente rielaborando il contenuto del libro, ignorando parti non utili ai fini del nostro corso e in ogni caso avendo come guida il programma svolto a lezione dal prof.~Martorella.

La seconda parte di questi Appunti � stata ottenuta partendo dalle lezioni del prof.~Berizzi e dal materiale che lo stesso ha messo in rete. Tuttavia, ampio spazio � stato dato a integrazioni dal~\citep{b:luise} (per il capitolo~\ref{cha:processi-stocastici}) e dal~\citep{b:Dandrea} (per i restanti capitoli della seconda parte), nell'ottica di fornire un fondamentale impianto teorico agli argomenti trattati (totalmente assente nelle slide del professore).

Le \annotation[nolist]{annotazioni a margine}{Come questa!} indicano parti del testo che sarebbe meglio rivedere, o perch� possono contenere imprecisioni o perch� il contenuto non � espresso nel migliore dei modi. I capitoli e le sezioni contrassegnati con \textbf{un asterisco (*)} sono opzionali: il professore (Martorella o Berizzi) non li ha trattati esplicitamente a lezione. Infine, i capitoli e le sezioni contrassegnati con \textbf{due asterischi (**)} sono solamente delle bozze.


\vspace{\stretch{1}}
\begin{flushright}
Antonio Macr�\\
\vspace{1pt}
9 luglio 2009 --- 13 novembre 2009
\end{flushright}
\vspace{\stretch{4}}


\phantomsection
\pdfbookmark[0]{Nota ai posteri}{note}
\section*{Nota ai posteri}

Avendo finalmente passato l'esame (12 novembre 2009), non metter� pi� mano a questi Appunti. Due note a chi vorr� continuare il lavoro.

Vanno convertite in \LaTeX{} (con PSTricks) alcune immagini, come la figura a pagina~\pageref{fig:sistema-di-elaborazione-e-trasmissione} e le varie figure nel paragrafo~\ref{sec:relazione-tcf-tsf-e-formule-di-poisson} (da pag.~\pageref{sec:relazione-tcf-tsf-e-formule-di-poisson}).
Andrebbero approfonditi argomenti come l'autocorrelazione di segnali periodici (la sua \acsfont{TCF} � la \acsfont{DSP}?), il filtraggio di segnali campionati. Va riaggiustato l'appendice~\ref{cha:numeri-complessi}. Inoltre, potrebbe essere utile un'ulteriore operazione di \emph{refactoring} per il capitolo~\ref{cha:processi-stocastici}: nel paragrafo~\ref{sec:relazione-io-tra-le-statistiche} forse � possibile togliere qualcosa e scremare un po'; che fare (pagina~\pageref{sec:densita-spettrale-di-un-processo-ssl}) della \emph{densit� spettrale di potenza di un processo stazionario}? Il prof.~non ha detto praticamente nulla sui processi parametrici, che c'� da dire? Infine, gli altri capitoli della seconda parte sono sostanzialmente ancora delle bozze.


\clearpage

\markboth{Acronimi, Annotazioni e Bibliografia}{Acronimi, Annotazioni e Bibliografia}

\phantomsection
\pdfbookmark[0]{Acronimi}{acronimi}
\section*{Acronimi}
\begin{acronym}[ATCF]
\acro{TSF}{\textit{Trasformata Serie Di Fourier}}
\acro{ATSF}{\textit{Antitrasformata Serie Di Fourier}}
\acro{TCF}{\textit{Trasformata Continua Di Fourier}}
\acro{ATCF}{\textit{Antitrasformata Continua Di Fourier}}
\acro{SLS}{\textit{Sistemi Lineari Stazionari}}
\acro{BIBO}{\textit{Bounded Input Bounded Output}}
\acro{TFS}{\textit{Trasformata Di Fourier di una Sequenza}}
\acro{SSS}{\textit{Stazionario in Senso Stretto}}
\acro{SSL}{\textit{Stazionario in Senso Lato}}
\acro{PAM}{\textit{Pulse Amplitude Modulation}}
\acro{ISI}{\textit{InterSymbol Interference}}
\end{acronym}

\pdfbookmark[0]{Annotazioni}{annotazioni}
\listoftodos[Annotazioni]

\pdfbookmark[0]{\bibname}{bibliografia}
{\nocite{*}\small\bibliography{bibliografia}}
