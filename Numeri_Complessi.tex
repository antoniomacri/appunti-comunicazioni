\chapter{Richiami sui numeri complessi**}
\label{cha:numeri-complessi}


\begin{align*}
z=a+jb\qquad \begin{array}{l}
a=\Re{\{z\}}:\text{parte reale}\\
b=\Im{\{z\}}:\text{parte immaginaria}
\end{array}\end{align*}

\begin{align*}
z=ce^{j\varphi}\qquad\begin{array}{l}
c=\text{modulo}\\
\varphi=\text{fase}
\end{array}\end{align*}

\begin{center}\framebox{\setlength{\unitlength}{1mm}
\begin{picture}(60,45)(-5,-5)
\put(-1,0){\vector(1,0){50}}
\put(0,-1){\vector(0,1){40}}
\put(-4,35){$\mathbf{Im}$}
\put(50,-1){$\mathbf{Re}$}
\put(39,-4){$a$}
\put(-4,29){$b$}
\put(18,16){$c$}
\put(12,3){$\varphi$}
\qbezier(10,0)(10,4)(8,6)
\multiput(40,-1)(0,4){8}{\line(0,1){2}}
\multiput(-1,30)(4,0){10}{\line(1,0){2}}
\thicklines
\put(0,0){\vector(4,3){40}}
\end{picture}}\end{center}

\begin{align*}\begin{array}{ll}
c=\sqrt{a^2+b^2} & a=c\cos\varphi\\
\varphi=\arctan\biggl(\frac{b}{a}\biggr) & b=c\sin\varphi
\end{array}\end{align*}

\begin{align*}
z=a+jb=c\cos\varphi +jc\sin\varphi=ce^{j\varphi}  \tag{$e^{j\varphi}=\cos\varphi+j\sin\varphi$}
\end{align*}

\begin{align*}
\Re{\{z\}}=a=\frac{1}{2}(z+z^*)\\
\Im{\{z\}}=b=\frac{1}{2}(z-z^*)
\end{align*}

\section{Operazioni algebriche}
\begin{align*}
 & z_1=a_1+jb_1=c_1e^{j\varphi_1}\\
 & z_2=a_2+jb_2=c_2e^{j\varphi_2}\\
 & z=z_1\pm z_2=(a_1\pm a_2)+j(b_1\pm b_2)\\
 & z=z_1\cdot z_2=(a_1a_2-b_1b_2)+j(a_1b_2-a_2b_1)=c_1c_2e^{j(\varphi_1+\varphi_2)}\\
 & |z|^2=z\cdot z^*=ce^{j\varphi}\cdot ce^{-j\varphi}=c^2e^{j(\varphi-\varphi)}=c^2\\
 & z^2=z\cdot z=ce^{j\varphi}\cdot ce^{j\varphi}=c^2e^{j2\varphi}=(a^2-b^2)+j2ab\\
 & z=\frac{z_1}{z_2}=z_1z_2^{-1}=\frac{c_1e^{j\varphi_1}}{c_2e^{j\varphi_2}}=\frac{c_1}{c_2}e^{j(\varphi_1-\varphi_2)}
\end{align*}

\section{Funzioni complesse di variabile reale}
\begin{align*}
 & z(t)\text{,}\qquad z\in\mathbb{C}\text{, }t\in\mathbb{R}\\
 & z(t)=a(t)+jb(t)=c(t)e^{j\varphi(t)}\\
 & \int_a^b z(t)\ud t=\int_a^b a(t)\ud t+j\int_a^b b(t)\ud t\\
 & \frac{\ud}{\ud t}[z(t)]=\frac{\ud}{\ud t}[a(t)] + j\frac{\ud}{\ud t}[b(t)]
\end{align*}





